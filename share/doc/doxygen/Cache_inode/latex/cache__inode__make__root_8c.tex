\section{cache\_\-inode\_\-make\_\-root.c File Reference}
\label{cache__inode__make__root_8c}\index{cache_inode_make_root.c@{cache\_\-inode\_\-make\_\-root.c}}
Insert in the cache an entry that is the root of the FS cached. 

{\tt \#include \char`\"{}LRU\_\-List.h\char`\"{}}\par
{\tt \#include \char`\"{}log\_\-functions.h\char`\"{}}\par
{\tt \#include \char`\"{}Hash\-Data.h\char`\"{}}\par
{\tt \#include \char`\"{}Hash\-Table.h\char`\"{}}\par
{\tt \#include \char`\"{}fsal.h\char`\"{}}\par
{\tt \#include \char`\"{}cache\_\-inode.h\char`\"{}}\par
{\tt \#include $<$unistd.h$>$}\par
{\tt \#include $<$sys/types.h$>$}\par
{\tt \#include $<$sys/param.h$>$}\par
{\tt \#include $<$time.h$>$}\par
{\tt \#include $<$pthread.h$>$}\par
\subsection*{Functions}
\begin{CompactItemize}
\item 
cache\_\-entry\_\-t $\ast$ {\bf cache\_\-inode\_\-make\_\-root} (cache\_\-inode\_\-fsal\_\-data\_\-t $\ast$pfsdata, hash\_\-table\_\-t $\ast$ht, cache\_\-inode\_\-client\_\-t $\ast$pclient, fsal\_\-op\_\-context\_\-t $\ast$pcontext, cache\_\-inode\_\-status\_\-t $\ast$pstatus)
\end{CompactItemize}


\subsection{Detailed Description}
Insert in the cache an entry that is the root of the FS cached. 

\begin{Desc}
\item[Author:]\begin{Desc}
\item[Author]deniel \end{Desc}
\end{Desc}
\begin{Desc}
\item[Date:]\begin{Desc}
\item[Date]2005/11/28 17:02:26 \end{Desc}
\end{Desc}
\begin{Desc}
\item[Version:]\begin{Desc}
\item[Revision]1.12 \end{Desc}
\end{Desc}
{\bf cache\_\-inode\_\-make\_\-root.c}{\rm (p.\,\pageref{cache__inode__make__root_8c})} : Inserts in the cache an entry that is the root of the FS cached.

Definition in file {\bf cache\_\-inode\_\-make\_\-root.c}.

\subsection{Function Documentation}
\index{cache_inode_make_root.c@{cache\_\-inode\_\-make\_\-root.c}!cache_inode_make_root@{cache\_\-inode\_\-make\_\-root}}
\index{cache_inode_make_root@{cache\_\-inode\_\-make\_\-root}!cache_inode_make_root.c@{cache\_\-inode\_\-make\_\-root.c}}
\subsubsection{\setlength{\rightskip}{0pt plus 5cm}cache\_\-entry\_\-t$\ast$ cache\_\-inode\_\-make\_\-root (cache\_\-inode\_\-fsal\_\-data\_\-t $\ast$ {\em pfsdata}, hash\_\-table\_\-t $\ast$ {\em ht}, cache\_\-inode\_\-client\_\-t $\ast$ {\em pclient}, fsal\_\-op\_\-context\_\-t $\ast$ {\em pcontext}, cache\_\-inode\_\-status\_\-t $\ast$ {\em pstatus})}\label{cache__inode__make__root_8c_a0}


cache\_\-inode\_\-make\_\-root: Inserts the root of a FS in the cache.

Inserts the root of a FS in the cache. This function will be called at junction traversal.

\begin{Desc}
\item[Parameters:]
\begin{description}
\item[{\em pfsdata}][IN] FSAL data for the root. \item[{\em ht}][IN] hash table used for the cache, unused in this call. \item[{\em pclient}][INOUT] ressource allocated by the client for the nfs management. \item[{\em pcontext}][IN] FSAL credentials. Unused here. \item[{\em pstatus}][OUT] returned status. \end{description}
\end{Desc}


Definition at line 116 of file cache\_\-inode\_\-make\_\-root.c.

References cache\_\-inode\_\-new\_\-entry().

Referenced by main().