\section{cache\_\-inode\_\-init.c File Reference}
\label{cache__inode__init_8c}\index{cache_inode_init.c@{cache\_\-inode\_\-init.c}}
Init the cache\_\-inode. 

{\tt \#include \char`\"{}LRU\_\-List.h\char`\"{}}\par
{\tt \#include \char`\"{}log\_\-functions.h\char`\"{}}\par
{\tt \#include \char`\"{}Hash\-Data.h\char`\"{}}\par
{\tt \#include \char`\"{}Hash\-Table.h\char`\"{}}\par
{\tt \#include \char`\"{}fsal.h\char`\"{}}\par
{\tt \#include \char`\"{}cache\_\-inode.h\char`\"{}}\par
{\tt \#include \char`\"{}stuff\_\-alloc.h\char`\"{}}\par
{\tt \#include $<$unistd.h$>$}\par
{\tt \#include $<$sys/types.h$>$}\par
{\tt \#include $<$sys/param.h$>$}\par
{\tt \#include $<$time.h$>$}\par
{\tt \#include $<$pthread.h$>$}\par
\subsection*{Functions}
\begin{CompactItemize}
\item 
hash\_\-table\_\-t $\ast$ {\bf cache\_\-inode\_\-init} (cache\_\-inode\_\-parameter\_\-t param, cache\_\-inode\_\-status\_\-t $\ast$pstatus)
\item 
int {\bf cache\_\-inode\_\-client\_\-init} (cache\_\-inode\_\-client\_\-t $\ast$pclient, cache\_\-inode\_\-client\_\-parameter\_\-t param, int thread\_\-index, void $\ast$pworker\_\-data)
\end{CompactItemize}


\subsection{Detailed Description}
Init the cache\_\-inode. 

\begin{Desc}
\item[Author:]\begin{Desc}
\item[Author]leibovic \end{Desc}
\end{Desc}
\begin{Desc}
\item[Date:]\begin{Desc}
\item[Date]2006/01/24 13:44:40 \end{Desc}
\end{Desc}
\begin{Desc}
\item[Version:]\begin{Desc}
\item[Revision]1.21 \end{Desc}
\end{Desc}
{\bf cache\_\-inode\_\-init.c}{\rm (p.\,\pageref{cache__inode__init_8c})} : Initialization routines for the cache\_\-inode.

Definition in file {\bf cache\_\-inode\_\-init.c}.

\subsection{Function Documentation}
\index{cache_inode_init.c@{cache\_\-inode\_\-init.c}!cache_inode_client_init@{cache\_\-inode\_\-client\_\-init}}
\index{cache_inode_client_init@{cache\_\-inode\_\-client\_\-init}!cache_inode_init.c@{cache\_\-inode\_\-init.c}}
\subsubsection{\setlength{\rightskip}{0pt plus 5cm}int cache\_\-inode\_\-client\_\-init (cache\_\-inode\_\-client\_\-t $\ast$ {\em pclient}, cache\_\-inode\_\-client\_\-parameter\_\-t {\em param}, int {\em thread\_\-index}, void $\ast$ {\em pworker\_\-data})}\label{cache__inode__init_8c_a1}


cache\_\-inode\_\-client\_\-init: Init the ressource necessary for the cache inode management on the client handside.

Init the ressource necessary for the cache inode management on the client handside.

\begin{Desc}
\item[Parameters:]
\begin{description}
\item[{\em pclient}][OUT] the pointer to the client to be initiated. \item[{\em param}][IN] the parameter for this cache client. \item[{\em thread\_\-index}][IN] an integer related to the 'position' of the thread, from 0 to Nb\_\-Workers -1\end{description}
\end{Desc}
\begin{Desc}
\item[Returns:]0 if successful, 1 if failed. \end{Desc}


Definition at line 147 of file cache\_\-inode\_\-init.c.

Referenced by main().\index{cache_inode_init.c@{cache\_\-inode\_\-init.c}!cache_inode_init@{cache\_\-inode\_\-init}}
\index{cache_inode_init@{cache\_\-inode\_\-init}!cache_inode_init.c@{cache\_\-inode\_\-init.c}}
\subsubsection{\setlength{\rightskip}{0pt plus 5cm}hash\_\-table\_\-t$\ast$ cache\_\-inode\_\-init (cache\_\-inode\_\-parameter\_\-t {\em param}, cache\_\-inode\_\-status\_\-t $\ast$ {\em pstatus})}\label{cache__inode__init_8c_a0}


cache\_\-inode\_\-init: Init the ressource necessary for the cache inode management.

Init the ressource necessary for the cache inode management.

\begin{Desc}
\item[Parameters:]
\begin{description}
\item[{\em param}][IN] the parameter for this cache. \item[{\em pstatus}][OUT] pointer to buffer used to store the status for the operation.\end{description}
\end{Desc}
\begin{Desc}
\item[Returns:]NULL if operation failed, other value is a pointer to the hash table used for the cache. \end{Desc}


Definition at line 117 of file cache\_\-inode\_\-init.c.

Referenced by main().