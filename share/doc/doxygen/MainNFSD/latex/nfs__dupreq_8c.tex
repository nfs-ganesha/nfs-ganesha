\section{nfs\_\-dupreq.c File Reference}
\label{nfs__dupreq_8c}\index{nfs\_\-dupreq.c@{nfs\_\-dupreq.c}}
{\tt \#include $<$stdio.h$>$}\par
{\tt \#include $<$sys/types.h$>$}\par
{\tt \#include $<$ctype.h$>$}\par
{\tt \#include $<$stdlib.h$>$}\par
{\tt \#include $<$dirent.h$>$}\par
{\tt \#include $<$netdb.h$>$}\par
{\tt \#include $<$netinet/in.h$>$}\par
{\tt \#include $<$arpa/inet.h$>$}\par
{\tt \#include $<$string.h$>$}\par
{\tt \#include $<$pthread.h$>$}\par
{\tt \#include $<$fcntl.h$>$}\par
{\tt \#include $<$sys/file.h$>$}\par
{\tt \#include $<$pwd.h$>$}\par
{\tt \#include $<$grp.h$>$}\par
{\tt \#include $<$rpc/rpc.h$>$}\par
{\tt \#include $<$rpc/svc.h$>$}\par
{\tt \#include $<$rpc/pmap\_\-clnt.h$>$}\par
{\tt \#include \char`\"{}LRU\_\-List.h\char`\"{}}\par
{\tt \#include \char`\"{}HashData.h\char`\"{}}\par
{\tt \#include \char`\"{}HashTable.h\char`\"{}}\par
{\tt \#include \char`\"{}log\_\-functions.h\char`\"{}}\par
{\tt \#include \char`\"{}nfs\_\-core.h\char`\"{}}\par
{\tt \#include \char`\"{}nfs23.h\char`\"{}}\par
{\tt \#include \char`\"{}nfs4.h\char`\"{}}\par
{\tt \#include \char`\"{}fsal.h\char`\"{}}\par
{\tt \#include \char`\"{}stuff\_\-alloc.h\char`\"{}}\par
{\tt \#include \char`\"{}nfs\_\-tools.h\char`\"{}}\par
{\tt \#include \char`\"{}nfs\_\-exports.h\char`\"{}}\par
{\tt \#include \char`\"{}nfs\_\-file\_\-handle.h\char`\"{}}\par
{\tt \#include \char`\"{}nfs\_\-dupreq.h\char`\"{}}\par
\subsection*{Functions}
\begin{CompactItemize}
\item 
unsigned int {\bf get\_\-rpc\_\-xid} (struct svc\_\-req $\ast$reqp)
\item 
int {\bf print\_\-entry\_\-dupreq} (LRU\_\-data\_\-t data, char $\ast$str)
\item 
int {\bf clean\_\-entry\_\-dupreq} (LRU\_\-entry\_\-t $\ast$pentry, void $\ast$addparam)
\item 
unsigned long {\bf dupreq\_\-value\_\-hash\_\-func} (hash\_\-parameter\_\-t $\ast$p\_\-hparam, hash\_\-buffer\_\-t $\ast$buffclef)
\item 
unsigned long {\bf dupreq\_\-rbt\_\-hash\_\-func} (hash\_\-parameter\_\-t $\ast$p\_\-hparam, hash\_\-buffer\_\-t $\ast$buffclef)
\item 
int {\bf compare\_\-xid} (hash\_\-buffer\_\-t $\ast$buff1, hash\_\-buffer\_\-t $\ast$buff2)
\item 
int {\bf display\_\-xid} (hash\_\-buffer\_\-t $\ast$pbuff, char $\ast$str)
\item 
int {\bf nfs\_\-Init\_\-dupreq} (nfs\_\-rpc\_\-dupreq\_\-parameter\_\-t param)
\item 
int {\bf nfs\_\-dupreq\_\-add} (long xid, struct svc\_\-req $\ast$ptr\_\-req, nfs\_\-res\_\-t $\ast$p\_\-res\_\-nfs, LRU\_\-list\_\-t $\ast$lru\_\-dupreq, dupreq\_\-entry\_\-t $\ast$$\ast$p\_\-dupreq\_\-pool)
\item 
nfs\_\-res\_\-t {\bf nfs\_\-dupreq\_\-get} (long xid, int $\ast$pstatus)
\item 
int {\bf nfs\_\-dupreq\_\-gc\_\-function} (LRU\_\-entry\_\-t $\ast$pentry, void $\ast$addparam)
\item 
void {\bf nfs\_\-dupreq\_\-get\_\-stats} (hash\_\-stat\_\-t $\ast$phstat)
\end{CompactItemize}
\subsection*{Variables}
\begin{CompactItemize}
\item 
nfs\_\-parameter\_\-t {\bf nfs\_\-param}
\item 
nfs\_\-function\_\-desc\_\-t {\bf nfs2\_\-func\_\-desc} [$\,$]
\item 
nfs\_\-function\_\-desc\_\-t {\bf nfs3\_\-func\_\-desc} [$\,$]
\item 
nfs\_\-function\_\-desc\_\-t {\bf nfs4\_\-func\_\-desc} [$\,$]
\item 
nfs\_\-function\_\-desc\_\-t {\bf mnt1\_\-func\_\-desc} [$\,$]
\item 
nfs\_\-function\_\-desc\_\-t {\bf mnt3\_\-func\_\-desc} [$\,$]
\item 
hash\_\-table\_\-t $\ast$ {\bf ht\_\-dupreq}
\end{CompactItemize}


\subsection{Function Documentation}
\index{nfs\_\-dupreq.c@{nfs\_\-dupreq.c}!clean\_\-entry\_\-dupreq@{clean\_\-entry\_\-dupreq}}
\index{clean\_\-entry\_\-dupreq@{clean\_\-entry\_\-dupreq}!nfs_dupreq.c@{nfs\_\-dupreq.c}}
\subsubsection[{clean\_\-entry\_\-dupreq}]{\setlength{\rightskip}{0pt plus 5cm}int clean\_\-entry\_\-dupreq (LRU\_\-entry\_\-t $\ast$ {\em pentry}, \/  void $\ast$ {\em addparam})}\label{nfs__dupreq_8c_0dd9a264f91e9823e807b34e6994f777}


clean\_\-entry\_\-dupreq: cleans an entry in the dupreq cache.

cleans an entry in the dupreq cache.

\begin{Desc}
\item[Parameters:]
\begin{description}
\item[{\em pentry}][INOUT] entry to be cleaned. \item[{\em addparam}][IN] additional parameter used for cleaning.\end{description}
\end{Desc}
\begin{Desc}
\item[Returns:]0 if ok, other values mean an error. \end{Desc}


Definition at line 227 of file nfs\_\-dupreq.c.\index{nfs\_\-dupreq.c@{nfs\_\-dupreq.c}!compare\_\-xid@{compare\_\-xid}}
\index{compare\_\-xid@{compare\_\-xid}!nfs_dupreq.c@{nfs\_\-dupreq.c}}
\subsubsection[{compare\_\-xid}]{\setlength{\rightskip}{0pt plus 5cm}int compare\_\-xid (hash\_\-buffer\_\-t $\ast$ {\em buff1}, \/  hash\_\-buffer\_\-t $\ast$ {\em buff2})}\label{nfs__dupreq_8c_1754ccd21821c745e816f32a7a459e24}


compare\_\-xid: compares the xid stored in the key buffers.

compare the xid stored in the key buffers. This function is to be used as 'compare\_\-key' field in the hashtable storing the nfs duplicated requests.

\begin{Desc}
\item[Parameters:]
\begin{description}
\item[{\em buff1}][IN] first key \item[{\em buff2}][IN] second key\end{description}
\end{Desc}
\begin{Desc}
\item[Returns:]0 if keys are identifical, 1 if they are different. \end{Desc}


Definition at line 364 of file nfs\_\-dupreq.c.\index{nfs\_\-dupreq.c@{nfs\_\-dupreq.c}!display\_\-xid@{display\_\-xid}}
\index{display\_\-xid@{display\_\-xid}!nfs_dupreq.c@{nfs\_\-dupreq.c}}
\subsubsection[{display\_\-xid}]{\setlength{\rightskip}{0pt plus 5cm}int display\_\-xid (hash\_\-buffer\_\-t $\ast$ {\em pbuff}, \/  char $\ast$ {\em str})}\label{nfs__dupreq_8c_30afd5b6d3dff5d4d7d4767089d59b44}


display\_\-xid: displays the xid stored in the buffer.

displays the xid stored in the buffer. This function is to be used as 'key\_\-to\_\-str' field in the hashtable storing the nfs duplicated requests.

\begin{Desc}
\item[Parameters:]
\begin{description}
\item[{\em buff1}][IN] buffer to display \item[{\em buff2}][OUT] output string\end{description}
\end{Desc}
\begin{Desc}
\item[Returns:]number of character written. \end{Desc}


Definition at line 384 of file nfs\_\-dupreq.c.\index{nfs\_\-dupreq.c@{nfs\_\-dupreq.c}!dupreq\_\-rbt\_\-hash\_\-func@{dupreq\_\-rbt\_\-hash\_\-func}}
\index{dupreq\_\-rbt\_\-hash\_\-func@{dupreq\_\-rbt\_\-hash\_\-func}!nfs_dupreq.c@{nfs\_\-dupreq.c}}
\subsubsection[{dupreq\_\-rbt\_\-hash\_\-func}]{\setlength{\rightskip}{0pt plus 5cm}unsigned long dupreq\_\-rbt\_\-hash\_\-func (hash\_\-parameter\_\-t $\ast$ {\em p\_\-hparam}, \/  hash\_\-buffer\_\-t $\ast$ {\em buffclef})}\label{nfs__dupreq_8c_cd77080a96d71b76efbdee9178171504}


dupreq\_\-rbt\_\-hash\_\-func: computes the rbt value for the entry in dupreq cache.

Computes the rbt value for the entry in dupreq cache. In fact, it just use the Xid itself (which is an unsigned integer) as the rbt value. This function is called internal in the HasTable\_\-$\ast$ function

\begin{Desc}
\item[Parameters:]
\begin{description}
\item[{\em hparam}][IN] hash table parameter. \item[{\em buffcleff\mbox{[}in\mbox{]}}]pointer to the hash key buffer\end{description}
\end{Desc}
\begin{Desc}
\item[Returns:]the computed rbt value.\end{Desc}
\begin{Desc}
\item[See also:]HashTable\_\-Init \end{Desc}


Definition at line 345 of file nfs\_\-dupreq.c.\index{nfs\_\-dupreq.c@{nfs\_\-dupreq.c}!dupreq\_\-value\_\-hash\_\-func@{dupreq\_\-value\_\-hash\_\-func}}
\index{dupreq\_\-value\_\-hash\_\-func@{dupreq\_\-value\_\-hash\_\-func}!nfs_dupreq.c@{nfs\_\-dupreq.c}}
\subsubsection[{dupreq\_\-value\_\-hash\_\-func}]{\setlength{\rightskip}{0pt plus 5cm}unsigned long dupreq\_\-value\_\-hash\_\-func (hash\_\-parameter\_\-t $\ast$ {\em p\_\-hparam}, \/  hash\_\-buffer\_\-t $\ast$ {\em buffclef})}\label{nfs__dupreq_8c_7363cd3531f6119e8a2047a57e51e08f}


dupreq\_\-rbt\_\-hash\_\-func: computes the hash value for the entry in dupreq cache.

Computes the hash value for the entry in dupreq cache. In fact, it just use the Xid modulo the hash array size. This function is called internal in the HasTable\_\-$\ast$ function

\begin{Desc}
\item[Parameters:]
\begin{description}
\item[{\em hparam}][IN] hash table parameter. \item[{\em buffcleff\mbox{[}in\mbox{]}}]pointer to the hash key buffer\end{description}
\end{Desc}
\begin{Desc}
\item[Returns:]the computed hash value.\end{Desc}
\begin{Desc}
\item[See also:]HashTable\_\-Init \end{Desc}


Definition at line 324 of file nfs\_\-dupreq.c.\index{nfs\_\-dupreq.c@{nfs\_\-dupreq.c}!get\_\-rpc\_\-xid@{get\_\-rpc\_\-xid}}
\index{get\_\-rpc\_\-xid@{get\_\-rpc\_\-xid}!nfs_dupreq.c@{nfs\_\-dupreq.c}}
\subsubsection[{get\_\-rpc\_\-xid}]{\setlength{\rightskip}{0pt plus 5cm}unsigned int get\_\-rpc\_\-xid (struct svc\_\-req $\ast$ {\em reqp})}\label{nfs__dupreq_8c_e04d38ace0d9a143801c1625135d4ad7}


get\_\-rpc\_\-xid: extract the transfer Id from a RPC request.

RPC Xid is used for RPC Reply cache. With UDP connection, the xid is in an opaque structure stored in xprt-$>$xp\_\-p2, but with TCP connection, the xid is in another opaque structure stored in xprt-$>$xp\_\-p1. xp\_\-p2 and xp\_\-p1 are mutually exclusive. The opaque structure are well defined in ONC RPC protocol definitions, and used internally by the ONC layers. Since I need to know the xid the structures are defined here.

\begin{Desc}
\item[Parameters:]
\begin{description}
\item[{\em reqp}]A pointer to the request to be examined.\end{description}
\end{Desc}
\begin{Desc}
\item[Returns:]the found xid. \end{Desc}


Definition at line 158 of file nfs\_\-dupreq.c.\index{nfs\_\-dupreq.c@{nfs\_\-dupreq.c}!nfs\_\-dupreq\_\-add@{nfs\_\-dupreq\_\-add}}
\index{nfs\_\-dupreq\_\-add@{nfs\_\-dupreq\_\-add}!nfs_dupreq.c@{nfs\_\-dupreq.c}}
\subsubsection[{nfs\_\-dupreq\_\-add}]{\setlength{\rightskip}{0pt plus 5cm}int nfs\_\-dupreq\_\-add (long {\em xid}, \/  struct svc\_\-req $\ast$ {\em ptr\_\-req}, \/  nfs\_\-res\_\-t $\ast$ {\em p\_\-res\_\-nfs}, \/  LRU\_\-list\_\-t $\ast$ {\em lru\_\-dupreq}, \/  dupreq\_\-entry\_\-t $\ast$$\ast$ {\em p\_\-dupreq\_\-pool})}\label{nfs__dupreq_8c_72a8bca40613bd2e5a727a7274ca8e66}


nfs\_\-dupreq\_\-add: adds an entry in the duplicate requests cache.

Adds an entry in the duplicate requests cache.

\begin{Desc}
\item[Parameters:]
\begin{description}
\item[{\em xid}][IN] the transfer id to be used as key \item[{\em pnfsreq}][IN] the request pointer to cache\end{description}
\end{Desc}
\begin{Desc}
\item[Returns:]DUPREQ\_\-SUCCESS if successfull\par
. 

DUPREQ\_\-INSERT\_\-MALLOC\_\-ERROR if an error occured during the insertion process. \end{Desc}


Definition at line 428 of file nfs\_\-dupreq.c.\index{nfs\_\-dupreq.c@{nfs\_\-dupreq.c}!nfs\_\-dupreq\_\-gc\_\-function@{nfs\_\-dupreq\_\-gc\_\-function}}
\index{nfs\_\-dupreq\_\-gc\_\-function@{nfs\_\-dupreq\_\-gc\_\-function}!nfs_dupreq.c@{nfs\_\-dupreq.c}}
\subsubsection[{nfs\_\-dupreq\_\-gc\_\-function}]{\setlength{\rightskip}{0pt plus 5cm}int nfs\_\-dupreq\_\-gc\_\-function (LRU\_\-entry\_\-t $\ast$ {\em pentry}, \/  void $\ast$ {\em addparam})}\label{nfs__dupreq_8c_35d644f1f1b2c333ddc223251d615205}


nfs\_\-dupreq\_\-gc\_\-function: Tests is an entry in dupreq cache is to be set invalid (has expired).

Tests is an entry in dupreq cache is to be set invalid (has expired).

\begin{Desc}
\item[Parameters:]
\begin{description}
\item[{\em pentry}][IN] pointer to the entry to test\end{description}
\end{Desc}
\begin{Desc}
\item[Returns:]1 if entry must be set invalid, 0 if not.\end{Desc}
\begin{Desc}
\item[See also:]LRU\_\-invalidate\_\-by\_\-function 

LRU\_\-gc\_\-invalid \end{Desc}


Definition at line 543 of file nfs\_\-dupreq.c.\index{nfs\_\-dupreq.c@{nfs\_\-dupreq.c}!nfs\_\-dupreq\_\-get@{nfs\_\-dupreq\_\-get}}
\index{nfs\_\-dupreq\_\-get@{nfs\_\-dupreq\_\-get}!nfs_dupreq.c@{nfs\_\-dupreq.c}}
\subsubsection[{nfs\_\-dupreq\_\-get}]{\setlength{\rightskip}{0pt plus 5cm}nfs\_\-res\_\-t nfs\_\-dupreq\_\-get (long {\em xid}, \/  int $\ast$ {\em pstatus})}\label{nfs__dupreq_8c_f44969aca0de15c4a2079988b6e6c266}


nfs\_\-dupreq\_\-get: Tries to get a duplicated requests for dupreq cache

Tries to get a duplicated requests for dupreq cache.

\begin{Desc}
\item[Parameters:]
\begin{description}
\item[{\em xid}][IN] the transfer id we are looking for \item[{\em pstatus}][OUT] the pointer to the status for the operation\end{description}
\end{Desc}
\begin{Desc}
\item[Returns:]the result previously set if $\ast$pstatus == DUPREQ\_\-SUCCESS \end{Desc}


Definition at line 502 of file nfs\_\-dupreq.c.\index{nfs\_\-dupreq.c@{nfs\_\-dupreq.c}!nfs\_\-dupreq\_\-get\_\-stats@{nfs\_\-dupreq\_\-get\_\-stats}}
\index{nfs\_\-dupreq\_\-get\_\-stats@{nfs\_\-dupreq\_\-get\_\-stats}!nfs_dupreq.c@{nfs\_\-dupreq.c}}
\subsubsection[{nfs\_\-dupreq\_\-get\_\-stats}]{\setlength{\rightskip}{0pt plus 5cm}void nfs\_\-dupreq\_\-get\_\-stats (hash\_\-stat\_\-t $\ast$ {\em phstat})}\label{nfs__dupreq_8c_ade2ed32dcdd7eb6990c8f99f5c81370}


nfs\_\-dupreq\_\-get\_\-stats: gets the hash table statistics for the duplicate requests.

Gets the hash table statistics for the duplicate requests.

\begin{Desc}
\item[Parameters:]
\begin{description}
\item[{\em phstat}][OUT] pointer to the resulting stats.\end{description}
\end{Desc}
\begin{Desc}
\item[Returns:]nothing (void function)\end{Desc}
\begin{Desc}
\item[See also:]HashTable\_\-GetStats \end{Desc}


Definition at line 570 of file nfs\_\-dupreq.c.\index{nfs\_\-dupreq.c@{nfs\_\-dupreq.c}!nfs\_\-Init\_\-dupreq@{nfs\_\-Init\_\-dupreq}}
\index{nfs\_\-Init\_\-dupreq@{nfs\_\-Init\_\-dupreq}!nfs_dupreq.c@{nfs\_\-dupreq.c}}
\subsubsection[{nfs\_\-Init\_\-dupreq}]{\setlength{\rightskip}{0pt plus 5cm}int nfs\_\-Init\_\-dupreq (nfs\_\-rpc\_\-dupreq\_\-parameter\_\-t {\em param})}\label{nfs__dupreq_8c_0e0c7a1fda2f7d756dd1b2f8b4f97db6}


nfs\_\-Init\_\-dupreq: Init the hashtable and LRU for duplicate request cache

Perform all the required initialization for hashtable and LRU for duplicate request cache

\begin{Desc}
\item[Parameters:]
\begin{description}
\item[{\em param}][IN] parameter used to init the duplicate request cache\end{description}
\end{Desc}
\begin{Desc}
\item[Returns:]0 if successful, -1 otherwise \end{Desc}


Definition at line 402 of file nfs\_\-dupreq.c.\index{nfs\_\-dupreq.c@{nfs\_\-dupreq.c}!print\_\-entry\_\-dupreq@{print\_\-entry\_\-dupreq}}
\index{print\_\-entry\_\-dupreq@{print\_\-entry\_\-dupreq}!nfs_dupreq.c@{nfs\_\-dupreq.c}}
\subsubsection[{print\_\-entry\_\-dupreq}]{\setlength{\rightskip}{0pt plus 5cm}int print\_\-entry\_\-dupreq (LRU\_\-data\_\-t {\em data}, \/  char $\ast$ {\em str})}\label{nfs__dupreq_8c_f812dda060fd0aa4f965adadb15dd22d}


print\_\-entry\_\-dupreq: prints an entry in the LRU list.

prints an entry in the LRU list.

\begin{Desc}
\item[Parameters:]
\begin{description}
\item[{\em data}][IN] data stored in a LRU entry to be printed. \item[{\em str}][OUT] string used to store the result.\end{description}
\end{Desc}
\begin{Desc}
\item[Returns:]0 if ok, other values mean an error. \end{Desc}


Definition at line 208 of file nfs\_\-dupreq.c.

\subsection{Variable Documentation}
\index{nfs\_\-dupreq.c@{nfs\_\-dupreq.c}!ht\_\-dupreq@{ht\_\-dupreq}}
\index{ht\_\-dupreq@{ht\_\-dupreq}!nfs_dupreq.c@{nfs\_\-dupreq.c}}
\subsubsection[{ht\_\-dupreq}]{\setlength{\rightskip}{0pt plus 5cm}hash\_\-table\_\-t$\ast$ {\bf ht\_\-dupreq}}\label{nfs__dupreq_8c_3ada0731d68a5b4aab865acfd3d73290}




Definition at line 140 of file nfs\_\-dupreq.c.\index{nfs\_\-dupreq.c@{nfs\_\-dupreq.c}!mnt1\_\-func\_\-desc@{mnt1\_\-func\_\-desc}}
\index{mnt1\_\-func\_\-desc@{mnt1\_\-func\_\-desc}!nfs_dupreq.c@{nfs\_\-dupreq.c}}
\subsubsection[{mnt1\_\-func\_\-desc}]{\setlength{\rightskip}{0pt plus 5cm}nfs\_\-function\_\-desc\_\-t {\bf mnt1\_\-func\_\-desc}[$\,$]}\label{nfs__dupreq_8c_331b43564c209696c670c70673d544e1}




Definition at line 224 of file nfs\_\-worker\_\-thread.c.\index{nfs\_\-dupreq.c@{nfs\_\-dupreq.c}!mnt3\_\-func\_\-desc@{mnt3\_\-func\_\-desc}}
\index{mnt3\_\-func\_\-desc@{mnt3\_\-func\_\-desc}!nfs_dupreq.c@{nfs\_\-dupreq.c}}
\subsubsection[{mnt3\_\-func\_\-desc}]{\setlength{\rightskip}{0pt plus 5cm}nfs\_\-function\_\-desc\_\-t {\bf mnt3\_\-func\_\-desc}[$\,$]}\label{nfs__dupreq_8c_7bcfef9b974d9b55a0f057d959a13bb8}




Definition at line 234 of file nfs\_\-worker\_\-thread.c.\index{nfs\_\-dupreq.c@{nfs\_\-dupreq.c}!nfs2\_\-func\_\-desc@{nfs2\_\-func\_\-desc}}
\index{nfs2\_\-func\_\-desc@{nfs2\_\-func\_\-desc}!nfs_dupreq.c@{nfs\_\-dupreq.c}}
\subsubsection[{nfs2\_\-func\_\-desc}]{\setlength{\rightskip}{0pt plus 5cm}nfs\_\-function\_\-desc\_\-t {\bf nfs2\_\-func\_\-desc}[$\,$]}\label{nfs__dupreq_8c_d1973964670037b91a33fd6b6e4c40bc}




Definition at line 169 of file nfs\_\-worker\_\-thread.c.\index{nfs\_\-dupreq.c@{nfs\_\-dupreq.c}!nfs3\_\-func\_\-desc@{nfs3\_\-func\_\-desc}}
\index{nfs3\_\-func\_\-desc@{nfs3\_\-func\_\-desc}!nfs_dupreq.c@{nfs\_\-dupreq.c}}
\subsubsection[{nfs3\_\-func\_\-desc}]{\setlength{\rightskip}{0pt plus 5cm}nfs\_\-function\_\-desc\_\-t {\bf nfs3\_\-func\_\-desc}[$\,$]}\label{nfs__dupreq_8c_177ddde86497df151d3b3ff5c876d90a}




Definition at line 191 of file nfs\_\-worker\_\-thread.c.\index{nfs\_\-dupreq.c@{nfs\_\-dupreq.c}!nfs4\_\-func\_\-desc@{nfs4\_\-func\_\-desc}}
\index{nfs4\_\-func\_\-desc@{nfs4\_\-func\_\-desc}!nfs_dupreq.c@{nfs\_\-dupreq.c}}
\subsubsection[{nfs4\_\-func\_\-desc}]{\setlength{\rightskip}{0pt plus 5cm}nfs\_\-function\_\-desc\_\-t {\bf nfs4\_\-func\_\-desc}[$\,$]}\label{nfs__dupreq_8c_de3381a451007489495bf0b92585520f}




Definition at line 218 of file nfs\_\-worker\_\-thread.c.\index{nfs\_\-dupreq.c@{nfs\_\-dupreq.c}!nfs\_\-param@{nfs\_\-param}}
\index{nfs\_\-param@{nfs\_\-param}!nfs_dupreq.c@{nfs\_\-dupreq.c}}
\subsubsection[{nfs\_\-param}]{\setlength{\rightskip}{0pt plus 5cm}nfs\_\-parameter\_\-t {\bf nfs\_\-param}}\label{nfs__dupreq_8c_eb8fc46586993cf210777049fca03969}




Definition at line 135 of file nfs\_\-init.c.