\section{fsal\_\-access.c File Reference}
\label{fsal__access_8c}\index{fsal\_\-access.c@{fsal\_\-access.c}}
FSAL access permissions functions.  


{\tt \#include \char`\"{}fsal.h\char`\"{}}\par
{\tt \#include \char`\"{}fsal\_\-internal.h\char`\"{}}\par
{\tt \#include \char`\"{}fsal\_\-convert.h\char`\"{}}\par
\subsection*{Functions}
\begin{CompactItemize}
\item 
fsal\_\-status\_\-t {\bf FSAL\_\-access} (fsal\_\-handle\_\-t $\ast$object\_\-handle, fsal\_\-op\_\-context\_\-t $\ast$p\_\-context, fsal\_\-accessflags\_\-t access\_\-type, fsal\_\-attrib\_\-list\_\-t $\ast$object\_\-attributes)
\end{CompactItemize}


\subsection{Detailed Description}
FSAL access permissions functions. 

\begin{Desc}
\item[Author:]\end{Desc}
\begin{Desc}
\item[Author]leibovic \end{Desc}
\begin{Desc}
\item[Date:]\end{Desc}
\begin{Desc}
\item[Date]2006/01/17 14:20:07 \end{Desc}
\begin{Desc}
\item[Version:]\end{Desc}
\begin{Desc}
\item[Revision]1.16 \end{Desc}


Definition in file {\bf fsal\_\-access.c}.

\subsection{Function Documentation}
\index{fsal\_\-access.c@{fsal\_\-access.c}!FSAL\_\-access@{FSAL\_\-access}}
\index{FSAL\_\-access@{FSAL\_\-access}!fsal_access.c@{fsal\_\-access.c}}
\subsubsection[{FSAL\_\-access}]{\setlength{\rightskip}{0pt plus 5cm}fsal\_\-status\_\-t FSAL\_\-access (fsal\_\-handle\_\-t $\ast$ {\em object\_\-handle}, \/  fsal\_\-op\_\-context\_\-t $\ast$ {\em p\_\-context}, \/  fsal\_\-accessflags\_\-t {\em access\_\-type}, \/  fsal\_\-attrib\_\-list\_\-t $\ast$ {\em object\_\-attributes})}\label{fsal__access_8c_a50fd5f13bbbdce0fd2279b34623b33a}


FSAL\_\-access : Tests whether the user or entity identified by the p\_\-context structure can access the object identified by object\_\-handle, as indicated by the access\_\-type parameter.

\begin{Desc}
\item[Parameters:]
\begin{description}
\item[{\em object\_\-handle}](input): The handle of the object to test permissions on. \item[{\em p\_\-context}](input): Authentication context for the operation (export entry, user,...). \item[{\em access\_\-type}](input): Indicates the permissions to be tested. This is an inclusive OR of the permissions to be checked for the user specified by p\_\-context. Permissions constants are :\begin{itemize}
\item FSAL\_\-R\_\-OK : test for read permission\item FSAL\_\-W\_\-OK : test for write permission\item FSAL\_\-X\_\-OK : test for exec permission\item FSAL\_\-F\_\-OK : test for file existence \end{itemize}
\item[{\em object\_\-attributes}](optional input/output): The post operation attributes for the object. As input, it defines the attributes that the caller wants to retrieve (by positioning flags into this structure) and the output is built considering this input (it fills the structure according to the flags it contains). Can be NULL.\end{description}
\end{Desc}
\begin{Desc}
\item[Returns:]Major error codes :\begin{itemize}
\item ERR\_\-FSAL\_\-NO\_\-ERROR (no error, asked permission is granted)\item ERR\_\-FSAL\_\-ACCESS (object permissions doesn't fit asked access type)\item ERR\_\-FSAL\_\-STALE (object\_\-handle does not address an existing object)\item ERR\_\-FSAL\_\-FAULT (a NULL pointer was passed as mandatory argument)\item Other error codes when something anormal occurs. \end{itemize}
\end{Desc}


Definition at line 57 of file fsal\_\-access.c.

References fsal2hpss\_\-testperm(), FSAL\_\-getattrs(), hpss2fsal\_\-error(), and TakeTokenFSCall().

Referenced by FSAL\_\-opendir().