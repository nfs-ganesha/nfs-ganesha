\documentclass[final]{ols}
\usepackage{url}
\usepackage{zrl}
\usepackage{framed}
\usepackage{color}
\definecolor{shadecolor}{rgb}{0.9,0.9,0.9}

\providecommand{\XFname}{OLS}
\providecommand{\XFuname}{papers2007}
\providecommand{\XFaddr}{linuxsymposium}
\providecommand{\XFyear}{2007}
\providecommand{\XFjwlA}{lockhart}
\providecommand{\XFjwlDomA}{redhat.com}
\providecommand{\XFjwlB}{jw.lockhart}
\providecommand{\XFjwlDomB}{comcast.net}
% Search for XFdep for conference/year changes

\begin{document}

% Required: Do not print the date.
\date{}

\title{Formatting Tips and Tricks} 
\subtitle{Some potentially helpful examples}

\author{
John W.\ Lockhart \\
{\em Red Hat, Inc.}\\
{\tt\normalsize {\XFjwlA}{@}{\XFjwlDomA}}\\
\and
Optional Second Author\\
{\em Second Institution}\\
{\tt\normalsize another@address.for.email.com}\\
} % end author section
\shortauthor{J.W.\ Lockhart}

\maketitle

% Required: Suppress page numbers on title page
\thispagestyle{empty}

\section*{Abstract}
This example paper contains tips and tricks to ensure that what you
write is what appears in the \textit{Proceedings} with as little
editing as possible.  The most important parts are at the end; please
read them.  (Okay, okay:  Section~\ref{lockhart-subrules} and
Figure~\ref{lockhart-fig1}.)

If you are new to {\LaTeX}, please read this paper in its entirety,
and check out its source and any other \texttt{.tex} files in the
\texttt{\small EXAMPLE} directory.  Reading \texttt{\small
EXAMPLE/EXAMPLE.tex} side by side with this PDF should be helpful.

%%% ******** XFdep: year
If you have a paper from the Linux Symposium or GCC Summit
(2002--2005), and would like to crib from its final formatting, please
drop me a note and I'll be happy to send along the edited source.
The 2006 papers and sources should be publicly and anonymously available at
\url{http://ols2006.108.redhat.com}. 
The final edited form of this year's source will appear at
\url{http://ols.108.redhat.com}. 

The tree was created based on the information on the conference
website.  If you don't have a subdirectory, create one along the same
lines.  Blank materials are in the \texttt{\small TEMPLATES}
directory; \texttt{ProtoMake} and \texttt{Blank.tex} are probably the
most interesting files.  Likewise, if your Abstract was available when
I looked, it has been included.  Feel free to edit it; it's just there
to get you started and to provide an example of how to properly
include files should you need to.

Please note that we generate both PDF and Postscript output for the
papers, so both \texttt{pdflatex} and \texttt{latex} will be used in the
process.  This happens automagically if you use the supplied build system;
you just need to ensure that both Postscript and PDF versions of your
graphics are available.  Note that using a non-Linux system is possible,
but will likely cause you additional work and possible compatibility
issues.  This is especially important when it comes to graphics and
embedded fonts.

\section{New This Year}
\begin{itemize}
\item Authors can submit papers and updates directly into the formatting
source control system.  See Figure~\ref{lockhart-fig-svn} and
\url{http://ols.108.redhat.com} for details; you will need an account on
\texttt{108} since the papers are not publicly available until after the
conference. 
\item The \texttt{108} site also hosts a number of mailing lists which are
useful for formatting questions and tips.
\item The current reference system for OLS builds is running Fedora
Core~6.  The \texttt{dviutils} package is required for full builds, and
may be found on the \texttt{108} website if needed.
\end{itemize}

\begin{figure*}[hbt]
\begin{shaded}
If you have created an account, sent mail to \texttt{lockhart} at
\texttt{redhat} dot \texttt{com} for access, and received a welcome note,
proceed as follows.  This example assumes that  your \texttt{108} username
is \texttt{itsme}, and avoids checking out all the papers in the tree.
\begin{center}
\begin{small}
\begin{verbatim}
# First check out the top-level directory
$ svn checkout https://ols.108.redhat.com/svn/ols/trunk/ols2007 \
  ols2007 --username itsme --non-recursive
$ cd ols2007
# Use the makefile to grab the rest of the templates
$ make update-templates
# Discover which directory has been set up for your paper.
# The tool is especially useful if your last name is Brown... :-)
$ ./Showtree.sh | grep -i MyLastName
# Then update to check out the skeletal paper.
$ svn update mylastname
$ cd mylastname
# Start editing.  To submit or update, just ensure that you're
# in the ols2007/mylastname directory, and then:
$ svn update
\end{verbatim}
\end{small}
\caption{Setting up and Using Subversion with OLS Papers}
\label{lockhart-fig-svn}
\end{center}
\end{shaded}
\end{figure*}

\section{Simple Formatting Tricks}

\LaTeX\ is just a fancy markup language\ldots \textit{most} of the
time.

Some of the more common font and layout conventions follow:
\begin{itemize}
\item \texttt{texttt} produces \texttt{typewriter} style.
\item \texttt{textit} produces \textit{italics}.
\item \texttt{textbf} produces \textbf{boldface}.
\item \texttt{textsc} produces \textsc{small caps}.
\item \texttt{\textit{Font}} \textbf{\textsc{styles}} can be
      \textit{\textbf{combined}}\footnote{Often eye-breakingly. Restraint is Good.}
\end{itemize}

Paragraphs
   can  be      awfully messy
in the source, and even
% what, a comment?
have comments interspersed.  Be careful with % unintentional
percent signs---75\% of the time you'll accidentally comment out the
rest of the text on the line.

Unescaped dollar signs will put you into math mode, so be likewise
careful.  Of course, that's sometimes exactly where you \textit{want}
to be.

Tildes do not produce tildes in \LaTeX ---think instead of
\textsc{html}'s \texttt{\&nbsp;} and you'll get the picture.  Instead,
you can use \texttt{{\textbackslash}{\~{}}\{\}} or
\texttt{{\textbackslash}textasciitilde} to produce a tilde.  
Table~\ref{lockhart-tab1} provides a list of characters that require
special handling.  Note that tables may ``float''---that is, {\LaTeX}
might move your table to a place where it all fits on a single page,
rather than putting it exactly where you have included it in your
source.  Be aware that it's easier to include references to tables and
figures than it is to force each into a particular position and adjust
the surrounding typesetting.
%
% that's 
%    \~{} 
% or
%    \textasciitilde
% for a tilde (without all the extra typesetting).
% Escape anything but a backslash by using a backslash.  Backslash
% itself is \textbackslash (as seen above).

\begin{table}[!th]
\begin{shaded}
\centering
\begin{small}
\begin{tabular}[b]{c|c|p{2.3cm}}
Char & Command & Otherwise \\
\hline
% #
\# & \texttt{{\textbackslash}\#} & argument number \tabularnewline
\hline
% $
\$  & \texttt{{\textbackslash}\$} & toggle math mode \tabularnewline
\hline
% %
\%  & \texttt{{\textbackslash}\%} & comment: ignore rest of line \tabularnewline
\hline
% &
\&  & \texttt{{\textbackslash}\&} & tabstop \tabularnewline
\hline
% _
\_  & \texttt{{\textbackslash}{\_}} & subscript in math mode \tabularnewline
\hline
% {
\{ & \texttt{{\textbackslash}\{} & open environment \tabularnewline
\hline
% }
\} & \texttt{{\textbackslash}\}} & close environment \tabularnewline
\hline
% ~
{\~{}}       & \texttt{{\textbackslash}{\~{}}\{\}} & non-breaking space \tabularnewline
{\textasciitilde}       & \texttt{{\textbackslash}textasciitilde} & non-breaking space \tabularnewline
\hline
% \
{\textbackslash} & \texttt{{\textbackslash}textbackslash} & begin command \tabularnewline
\end{tabular}
\end{small}
\caption{{\LaTeX} characters that require special handling}
\label{lockhart-tab1}
\end{shaded}
\end{table}

\subsection{New Macros}\label{lockhart-newmacros}

A number of macros based on the \texttt{url} package are available
for this year.  They are:
\begin{itemize}
\item \ident{ident} -- intended for identifiers,
  \texttt{{\textbackslash}ident\{some\_text\}} sets the text in
  \texttt{tt} and may break the line at any punctuation.  Spaces are deleted.
\item \ident{lident} -- intended for long identifiers, this works the
  same as \ident{ident}, but sets the text in a smaller font.
\item \ident{code} -- intended for short excerpts of code, this works
  like \ident{ident}, except that spaces are preserved.  Lines are not
  broken on spaces.
\item \ident{lcode} -- intended for longer excerpts of code, this works
  like \ident{code}, except that text is set in a smaller font.  This
  probably does not work correctly for multi-line code fragments;
  consider using the \texttt{cprog} package for that.
\item \ident{brcode} -- intended for excerpts of source code, this works
  like \ident{code}, except that line breaks may occur at spaces.
\item \ident{lbrcode} -- intended for excerpts of source code, this works
  like \ident{brcode}, except that text is set in a smaller font.
\end{itemize}

Examples are shown in Table~\ref{lockhart-macro-examples}.

\begin{table*}[tb]
\begin{shaded}
\begin{itemize}
\item \verb|\ident{a_long_identifier}| --- this example in turn yields \ident{a_long_identifier}

\item \texttt{{\textbackslash}lident|An\_Even Lon ger Identifier|} --- this
  in turn
  yields \lident|An_Even Lon ger Identifier|

\item \verb|\lcode{int un_useful(int *a) { return *a; }}| --- this
  yields
  \lcode{int un_useful(int *a) { return *a; }}

\item \verb|\lbrcode{int un_useful(int *a) { return *a; }}| --- this
  yields
  \lbrcode{int un_useful(int *a) { return *a; }}

\end{itemize}
\caption{Examples of New Macros}
\label{lockhart-macro-examples}
\end{shaded}
\end{table*}

Many thanks go to Zack Weinberg for studying prior years' templates and
proceeding to write the \texttt{ols.cls} class and other crucial bits of
infrastructure for 2005.  We continue to use this system, with a few minor
tweaks to its features.  So if you started writing using last year's
templates, you should be fine with this set.

The document class for your paper can be one of \texttt{final},
\texttt{galley}, or \texttt{draft}.  This year's default is
\texttt{final}, since the \texttt{galley} option confused more people
than it helped.  Please be sure to submit your paper using the
\texttt{final} class.  The differences are:

\begin{itemize}

\item \texttt{galley} --- All ``this doesn't fit'' warnings are
suppressed, and references are printed as textual keys rather than as
numbers.  This is because when you're working on the paper and have
the TeX in front of you along with a printout, it's often easier to
use ``See Section [whatever-I-called-it]'' than ``See Section 8.''
That way, you can just search the .tex file for the section by name to
make changes, since there's no ``Section 8'' in said \texttt{.tex}
file\ldots

\item \texttt{proof} --- All ``this doesn't fit'' warnings are active,
as are references.  Overfull hboxes make ugly black blobs in the
margin.  You can use this mode to tidy up formatting after you're done
writing.  This mode is the same as the article class's ``draft'' mode.

\item \texttt{final} --- All warnings and references are active, and
the paper produced will be similar to the one which will be
published (but without headers, final page numbers, and such).
\end{itemize}

\section{Typesetting conventions}

You shouldn't have to worry too much here, but I'll illustrate a few
things.

Quotation marks, both `single' and ``double,'' look good in body text,
while other \texttt{"styles"} might look better for other uses.  Note
that when you're typesetting for a compiler, punctuation goes outside
the \texttt{"quotation marks",} but punctuation is placed
\textit{inside} the quotation marks for ``narrative.''
If you are one of those who like to introduce a bunch of ``terms,'' each
one in ``quotes,'' please use italics for the first instance of each
\textit{term} instead. 

There are multiple flavors of dashes---the em dash, the en--dash, the
oft-used hyphen, and the minus sign (math mode: $2x - 3$).  Note that
the preceding sentence contains them all.

\subsection{Choices for uniformity}

For source code, we have chosen the common style of not beginning a
line with a comma.  The compiler doesn't care, but keeping the printed
page consistent between papers is useful.

%% The \linebreak[0] command can be used, but it's very very painful.
%% See the "New Macros" section for a better alternative.
Identifiers may need to be split between lines, so we use a typewriter font
and mark up the string appropriately:
\texttt{sys\_\linebreak[0]sched\_\linebreak[0]yield()} or
\texttt{\small A\_\linebreak[0]REALLY\_\linebreak[0]LONG\_\linebreak[0]IDENTIFIER\_\linebreak[0]THAT\_\linebreak[0]NEEDS\_\linebreak[0]TO\_\linebreak[0]BE\_\linebreak[0]THIS\_\linebreak[0]LONG}
would be good examples.\footnote{Alternatively, see the macros in
Section~\ref{lockhart-newmacros}.}  To tell {\LaTeX} that an unhyphenated line
break is okay if required, just use \texttt{{\textbackslash}linebreak[0]}.

\subsection{Points of English}

A few nitpicks:
\begin{enumerate}
\item \textit{it's} is a macro which expands to \textit{it is}.  It
      has no other meaning. 
\item \textit{its} is possessive.
\item Items in a series are:  \textit{a}, \textit{b}\textbf{,} and \textit{c}.  
      Never \textit{a}, \textit{b} and \textit{c}.  This rule makes it
      much simpler when you must use complex values of (for example)
      \textit{b}.  For truly long constructs, you may use a semicolon
      as a delimiter rather than a comma.
\item Some phrases should be hyphenated---for instance, when you're
  using an adjective to modify another adjective, or a noun that
  appears before another.  A high-performance system; a win-win
  situation; a high-level loop transformation; a slow-moving train,
  but a slowly moving car; that sort of thing.  Most of the time,
  people will still be able to parse the results easily if the sentence isn't
  perfect. 
\item Be happy, know your homonyms.  There, they're, their.  To, two,
      too.  Your, you're.  And so forth.  Spelling checkers show their
      limitations on this\ldots
\end{enumerate}

Of course, proofreading is a wonderful thing, and every bit of it you
(or any guinea pigs you can persuade) do is a Good Thing.  I'll
correct what I notice, but I have only two eyes and there's a lot of
margin-crunching formatting to be done.  There are certain
times, often with non-native speakers, where I'm not clear on the
meaning.  If I catch something like that in time, I'll ask; if not,
chances are that I'll keep my hands off of the section in question so
as not to insert a woefully incorrect meaning. 

\section{Tools}

It helps to have the following installed on your system:
\begin{itemize}

\item \textbf{\texttt{tetex}}.  The most common \TeX\ package for Linux.
  Related useful packages include \textbf{\texttt{dviutils}}, \textbf{\texttt{xdvi}},
  \textbf{\texttt{dvips}}, and \textbf{\texttt{ghostscript}}.  Be aware that
  certain distributions package \texttt{pdflatex} separately.

\item \textbf{\texttt{transfig}}.  Graphics in \texttt{.fig} format,
  useful for figures. 

\item \textbf{\texttt{dia}}.  Useful for figures.  Other commonly used
  graphics programs include \textbf{\texttt{transfig}}, \textbf{\texttt{xfig}},
  \textbf{\texttt{inkscape}}, and \textbf{\texttt{OpenOffice.org}}
  \textbf{\texttt{transfig}}.  Please note that your export options are
  crucial, and 
  that we recommend sending along the original, native file as well.
  You should aim for portability: for instance, certain OpenOffice
  versions let you choose between EPS with Pango fonts, or EPS --- you
  want the non-Pango EPS for portable printing.

\item \textbf{\texttt{ImageMagick}}.  Great for photographs and graphics
  manipulation \& conversion, especially the \texttt{convert} program.

\item \textbf{\texttt{xpdf}}, \textbf{\texttt{evince}}, \textbf{\texttt{kpdf}}, or
  \textbf{\texttt{acroread}} for viewing PDF files.  Other viewers can also do
  a nice job. 

\item \textbf{\texttt{pdftk}} if you wish to concatenate PDF pages or
  perform several other tasks with PDFs.  This package is required for
  building the final Proceedings.

\item \textbf{\texttt{gnuplot}} for drawing graphs based on large
  quantities of numeric data.

\item \textbf{\texttt{dviconcat}} is required for building the entire
  Proceedings, and can be had from the \textit{dviutils} package,
  available on \url{http://ols.108.redhat.com} or available as source,
  Debian, or SuSE packages (possibly other distros).

\item \textbf{\texttt{emacs}} is my editor of choice, in part due to its
``LaTeX Fill'' mode, which does a very nice job of handling markup and
syntax, with automatic line wraps.  Please check to see if your editor of
choice has a mode for editing {\LaTeX} files; it could save you from
tracking down syntax errors and such.  (I also use and enjoy
\texttt{gvim}, especially its \textit{diff} mode.)

\end{itemize}

Beware any program that cannot export structured graphics, or has
troubles respecting standards.  These are most commonly found on
non-Linux systems; with some of these, you may be better off
printing to a hypothetical Apple Laserwriter on \texttt{FILE:} and
converting the proceeds than in trying to rely on the program doing
anything remotely sane and portable.

\section{Examples}

Some examples from previous conferences have been included
in this package; hopefully they'll be useful in handling code
examples.  Reducing everything to \texttt{footnotesize} or setting it
\texttt{verbatim} won't magically make it fit on the page, alas.  Have
a look in the \texttt{EXAMPLE} directory to find these items:
\begin{itemize}
\item {\raggedright \texttt{\small bibli\-og\-raphy.tex}, \texttt{\small bibli\-og\-ra\-phy2.tex}, and
  \texttt{\small ref\-er\-ences.tex}.  Different ways of citing any relevant
  works external to your paper.}
\item \texttt{conditional.tex}.  If you have {\LaTeX} code that works
  only by itself and need to do conditional processing, here's an example.
\item \texttt{\small complexCode/complexFigure.tex}.  An example of a complex
  figure containing side-by-side C code.
\item \texttt{figures.tex}.  Different ways of doing figures.
\item \texttt{includegraphics.tex}.  Different ways to include graphics.
\item \texttt{legalese.tex}.  Legal disclaimers.
\item \texttt{multipleAuthors.tex}.  Formatting examples for multiple authors.
\item \texttt{tables.tex}.  Different ways to do tables.
\end{itemize}

\subsection{Bad Examples}

A prior year's paper gave the example of setting \texttt{verbatim}
sections in \texttt{tt}.  Repetitiously and redundantly enough, that's
the default.  So, please, no instances of
\begin{verbatim}
  {\tt
  \begin{verbatim}
   ...
\end{verbatim}

\begin{small}
\centering
\textbf{Corrected.}  You might, however, wish to do something like this instead:
\begin{verbatim}
  \begin{small}
  \centering
  \textbf{Corrected.}  You ...
  \begin{verbatim}
    ...
\end{verbatim}
\end{small}
Of course, check the source of this document
(\lident{EXAMPLE/myPaper.tex}) for more ideas.  Valid font sizes, for
instance, include \texttt{normalsize}, \texttt{small},
\texttt{footnotesize}, \texttt{scriptsize}, and \texttt{tiny}.  Please
don't use anything larger than \texttt{normalsize}.


Another extant bad example is the practice of ending paragraphs with a
double backslash (\texttt{\textbackslash\textbackslash}) \textit{and}
a blank line.  This creates unwanted, superfluous whitespace between
paragraphs.  \LaTeX\ is, believe it or not, supposed to be easy.  Just
leave one or more blank lines between paragraphs and you'll be fine.


\section{Style packages}

You
will find some additional useful packages in the \texttt{Texmf}
directory.  The empty papers are set up to use the
\texttt{url}, \texttt{zrl}, and \texttt{graphicx} packages by default,
in hopes that this will be useful for most papers.

You may also find it helpful to set the \texttt{TEXINPUTS} environment
variable as follows:
\begin{center}
{\footnotesize \texttt{export TEXINPUTS='.//:\$\{LOCALTEX\}//:'}}
\end{center}
%
% or for those of you who'd like to cut'n'paste from the source:
% export TEXINPUTS='.//:${LOCALTEX}//:'
%
Adding the above to your \texttt{\textasciitilde/.bashrc} can
save you the trouble of typing it for future runs.  The build system
uses this setup by default.

To build your paper, you should be able to \texttt{cd} to the toplevel
directory (the one that contains your individual directory) 
and type the following at a shell prompt:

\begin{small}
\begin{verbatim}
 DIRS=yourname make
\end{verbatim}
\end{small}

Ambitious authors are encouraged to install the \texttt{dviutils}
and \texttt{pdftk} packages and type \texttt{make} from the top-level directory.
If all goes well, you'll get something that looks quite like the finished \textit{Proceedings}.

\section{Graphics and Symbols}

For importing graphics, don't forget to omit any file extensions.
That's because \texttt{latex} and \texttt{pdflatex} look for
different formats.  The output formats we generate are PDF, PS, and
DVI; you will thus want to generate both EPS and PDF copies of any
figures that use structured graphics.

The easiest ways to get special symbols such as
Registered\textregistered\ and Trademark\texttrademark\ 
is to use the \LaTeX2e\ \texttt{{\textbackslash}text} constructs:
thus, \texttt{{\textbackslash}textregistered} and 
\texttt{{\textbackslash}texttrademark}.

We generally try to leave a small margin around each figure, and thus
use constructs such as those in Figure~\ref{lockhart-fig-igfx} for a
full-width or full-column picture. 

\begin{figure*}[htb]
\begin{shaded}
\begin{center}
\begin{small}
\begin{verbatim}
% By omitting the extension,
%  - pdflatex finds jwl-page-fig.pdf and jwl-col-fig.pdf
%  - latex finds jwl-page-fig.eps and jwl-col-fig.eps

% Full pagewidth figure, spanning both columns:
\begin{figure*}
\includegraphics[width=0.9\textwidth]{jwl-page-fig}
\caption{The caption appears beneath the figure}
\label{jwl-page-fig-label}
\end{figure*}

% Single-column figure:
\begin{figure}
\includegraphics[width=0.9\columnwidth]{jwl-col-fig}
\end{figure}
\end{verbatim}
\end{small}
\caption{How to use \texttt{includegraphics}}
\label{lockhart-fig-igfx}
\end{center}
\end{shaded}
\end{figure*}

Please note that the \texttt{.eps} and \texttt{.pdf} extensions have been omitted.
latex will automatically search for the former; pdflatex, the latter.
Adding an extension will break one tool or the other.  For the full
page width, use \texttt{figure*}; otherwise, use \texttt{figure}.  See
Figure~\ref{lockhart-fig-igfx} for the example.

This year's Proceedings will be in color (or, if you prefer, 
\textit{colour}).  If you choose to use color graphics and figures,
please do so effectively, drawing attention to important information
and distinctions in your paper.  A single box or letter doesn't
usually add much information, but does squander resources needlessly.

Please avoid using bitmapped graphics unless you have no other choice
(for instance, a JPG photograph).  Structured graphics always produce
better printed output.

\section{\TeX\ References}

If you aren't familiar with {\LaTeX}, there are many sources of
information available.  Your distribution might have additional
documentation in \brcode{/usr/share/texmf}, or you might find manuals
for a package (such as \texttt{cprog}) at {\small\url{http://www.ctan.org}}.

If you are completely new to {\TeX} and {\LaTeX}, you will probably
find it highly useful to visit \texttt{\small http://www.tug.org/} and
especially \texttt{\small http://www.tug.org/begin.html} for online
and paper references.

For a free and extremely useful document, try:
\texttt{\small http://www.tug.org\linebreak[0]/tex-archive\linebreak[0]/info\linebreak[0]/lshort\linebreak[0]/english\linebreak[0]/lshort.pdf}.  
Note that translations\footnote{French, for instance:
\url{http://www.tug.org/tex-archive/info/lshort/french/flshort-3.20.pdf};
note also that this section of the Example paper shows different ways
of handling URLs.}
are available, for those more comfortable in something other than
English: 
\texttt{\small http://www.tug.org\linebreak[0]/tex-archive\linebreak[0]/info\linebreak[0]/lshort/}

%%% Cut'n'paste versions of those URLs:
% http://www.tug.org/tex-archive/info/lshort/english/lshort.pdf
% http://www.tug.org/tex-archive/info/lshort/french/flshort-3.20.pdf
% http://www.tug.org/tex-archive/info/lshort/

I tend to use \textit{A Guide to \LaTeX} (Kopka \& Daly, ISBN 0-201-39825-7) and the
\textit{\LaTeX\ Graphics Companion} (Goossens, Rahtz, \& Mittelbach)
the most these days.

You are also welcome to send questions to me at
\texttt{{\XFjwlA}{@}{\XFjwlDomA}} (work) or
\texttt{{\XFjwlB}{@}{\XFjwlDomB}} (home).
%
% {}'s begin a new environment in TeX, as in C.
% A few extra {}'s might let an email address escape notice 
% by spammers' collecting 'bots, should the .tex file wind 
% up on a website somewhere at some point.
%

As usual, please refrain from submitting anything remotely resembling
a Microsoft Word \texttt{.doc} file\ldots \texttt{<grimace>}.  It's a
\textit{lot} easier for me to fix up plain ASCII text and
convert/insert accompanying graphics, if you find yourself terminally
confused or in a dire emergency.

\begin{figure}[!ht]
\begin{shaded}
\begin{center}
\hrule
\vspace*{2mm}
\textbf{\textsc{Submitting a Paper}}
\begin{footnotesize}
%% **************** XFdep ******************
\begin{verbatim}
  cd ols2007
  make clean
  tar zcf yourLastName.tar.gz \
      yourLastName
\end{verbatim}

E-mail the resulting tarball to
\texttt{{\XFuname}{@}{\XFaddr}{.}org}.
\end{footnotesize}
\vspace*{2mm}
\hrule

\caption{Submitting a paper}
\label{lockhart-fig1}

\end{center}
\end{shaded}
\end{figure}

\section{Build Issues}

\subsection{PDF from EPS}

To make PDF graphics from your EPS files, you can adjust your
\texttt{Makefile.inc} to run \texttt{epstopdf} (which is generally part of the
\texttt{tetex} package on most distros) automatically.  Here's an
approach that would work for an author named \texttt{auth}\ldots

Modify \texttt{\small{\XFname}{\XFyear}/author/Makefile.inc} in the following fashion
(keeping in mind that those aren't leading spaces, but tabs):

\begin{scriptsize}
\begin{verbatim}
## Add any additional .tex or .eps files below:
auth/auth.dvi author/auth-proc.dvi: \
    auth/auth.tex author/auth-abstract.tex \
    auth/auth-figure1.eps \
    auth/auth-figure1.pdf \
    auth/auth-figure2.eps \
    auth/auth-figure2.pdf 

auth/auth-%.pdf: auth/auth-%.eps
    epstopdf $<
\end{verbatim} %$
\end{scriptsize}

Doing a \texttt{make} on the toplevel will get you an
\texttt{author-proc.pdf} file, which is pretty much how it'll look in
the finished Proceedings.  Feel free to delete the other authors'
directories in your tree.


\section{Simple rules to keep your formatting team happy}
\label{lockhart-subrules}
\begin{enumerate}
\item To submit your paper, either use SVN, or just \texttt{make clean} in your
  directory, \texttt{tar} it up, and send the resulting gzipped tarball to
  \texttt{{\XFuname}{@}{\XFaddr}.org}.
  See Figure~\ref{lockhart-fig1} for an example.
\item Updates.  If you need to change something, please send both
  a patch and an updated tarball.  The most convenient form depends on
  how many changes have been made since you submitted your paper.
  However, if your change is trivial---a line or two, for instance---a
  simple email will do.  
\item Use the existing directory structure, please.  The directory
  names are intended to be the last name of the presenter (lowercase,
  punctuation omitted); the main paper should be
  \texttt{lastname.tex} and any additional files should be
  \texttt{lastname-file.extension}.   This is basically to keep
  the file owners straight, and to allow us the option to
  instruct {\LaTeX} to search the entire (sub)directory hierarchy for
  input files.  You don't want someone else's file by mistake, right?
  Putting your name on it helps to keep things straight.  The same
  goes for \verb|\label{}| and \verb|\ref{}| commands.
\item Omit file extensions and pathnames in your {\LaTeX} source,
  please.  By omitting the path and just saying
  \texttt{{\textbackslash}input\{lockhart-abstract\}}, 
  a paper can be built from both its directory and from its
  parent directory.  For graphics, omitting the extension lets
  \texttt{latex} or 
  \texttt{pdflatex} pick its preferred input format for the best
  possible results.
\item No proprietary document/graphics formats, please.  This
  especially means MS 
  Office, Visio, or other such tools.  \LaTeX\ can, however, import
  EPS and PDF, if you can save in those formats.  Although you can convert
  a bitmap into EPS/PDF, please don't do so.  If you have screenshots or
  photos, PNG or JPG can be used.  Everything else should be structured
  graphics only.
\item Originals, please.  For example, if you have photographs, send
  along the full-resolution JPG (crop out any undesired elements if
  necessary, but use the maximum resolution).  For diagrams, please
  send the XFig or Dia files. 
  This ensures the best possible print quality.  Printing will be in
  black and white, but the online PDF's will be in full color.  Your
  screen is probably about 72dpi, but the typesetter is probably using
  something that's at least 1200dpi.  The more resolution, the better.
  (If, however, your originals are outrageously huge, feel free to ask!)
  Since hardcopy will be printed in Ottawa, the papersize will be
  North American ``letter.''  Please keep that in mind if you are
  concerned about page breaks and such.
\item Do \textbf{\textit{not}} use sans-serif fonts, or go changing
  global font sizes.  We're using 12-point Times Roman for body text.
  Likewise, please don't go haywire with italics.  I once received a
  huge collection of tables, each of which set the font size and face
  on an item-by-item basis.  \textit{Incorrectly}.  
\item Those of you who like to begin lines of code with commas:  as
  previously mentioned, we're
  typesetting the code with the comma attached to the preceding
  identifier (as most publishers do).  Feel free to post your
  preferred version to the web and to refer to it in the paper.
\item If possible, please avoid trivial new macros.  Should you need
  to add something, though, please use
  \texttt{{\textbackslash}providecommand} rather than
  \texttt{{\textbackslash}newcommand}, and try for a relatively
  unique name (papers tend to blur together during long editing sessions).
\item Trivia note:  generally speaking, it takes longer to edit a
  submission from a {\TeX}spert than plain, unmarked ASCII.  If you
  consider yourself a {\LaTeX} expert and love to write fancy new
  commands, please consider contributing clean-ups or well-tested
  new features for the infrastructure rather than customizing the
  daylights out of your submission.  Thanks!
\end{enumerate}

This paper builds correctly on Fedora Core 6 and Red Hat Enterprise Linux
5.  Other distributions haven't been tested, but should work.  If you run
into problems, please let me know so that I can try to make it work.

And remember, it's only typesetting, not rocket science.  Or hacking
compilers or kernels.  \texttt{:-)}  Have some fun along the way\ldots

\end{document}
